\documentclass{beamer}
\usetheme{Madrid}
\usecolortheme{default}
\usepackage{graphicx}
\usepackage{amsmath}
\usepackage{booktabs}
\usepackage{multicol}

% Set graphics path
\graphicspath{{../paper/figures/}}

\title[EU ETS Effects on Emissions]{Quantifying the Effects of Climate Policy Stringency on Verified Emissions and Satellite-Derived NOx}
\subtitle{Master's Thesis}
\author{Arnav Agrawal}
\institute{Columbia University \\ MA in Quantitative Methods in the Social Sciences}
\date{December 2025}

\begin{document}

%% ========================================================================
%% SLIDE 1: TITLE
%% ========================================================================
\begin{frame}
\titlepage
\end{frame}

%% ========================================================================
%% SLIDE 2: RESEARCH QUESTION
%% ========================================================================
\begin{frame}{Research Question \& Motivation}

\textbf{Core Question:} How does EU ETS policy stringency affect emissions at the facility level?

\vspace{0.3cm}
\textbf{Dual-Outcome Approach:}
\begin{itemize}
    \item \textbf{Verified CO$_2$}: Administrative data from EU ETS registry (gold standard)
    \item \textbf{Satellite NOx}: Independent physical measurement from TROPOMI
\end{itemize}

\vspace{0.3cm}
\textbf{Why Both?}
\begin{itemize}
    \item CO$_2$: Accurate compliance trajectories, global climate impact
    \item NOx: Local air quality co-benefits, health impacts (respiratory, cardiovascular)
    \item Cross-validation: Agreement between independent systems strengthens causal claims
\end{itemize}

\vspace{0.3cm}
\textbf{Sample:} 521 large combustion plants across 82 NUTS2 regions (2018--2023)

\end{frame}

%% ========================================================================
%% SLIDE 3: EU ETS BACKGROUND
%% ========================================================================
\begin{frame}{EU ETS Background}

\textbf{Cap-and-Trade System} (established 2005)
\begin{itemize}
    \item World's largest carbon market, covers $\sim$40\% of EU GHG emissions
    \item Large combustion plants ($\geq$20 MW) must hold allowances = verified emissions
\end{itemize}

\vspace{0.3cm}
\textbf{Treatment Variable: Allocation Ratio}
\[
R_{it} = \frac{\text{Free Allocation}_{it}}{\text{Verified Emissions}_{it}}
\]
\begin{itemize}
    \item $R < 1$: Shortfall $\rightarrow$ must purchase allowances (policy stringency)
    \item $R > 1$: Surplus $\rightarrow$ can sell excess allowances
\end{itemize}

\vspace{0.3cm}
\textbf{Key Policy Evolution (Phase III/IV):}
\begin{itemize}
    \item Power sector lost most free allocation $\rightarrow$ full marginal carbon cost
    \item Electricity: mean $R = 0.52$ vs. Other sectors: mean $R = 1.03$
\end{itemize}

\end{frame}

%% ========================================================================
%% SLIDE 4: TREATMENT VARIATION
%% ========================================================================
\begin{frame}{Policy Stringency: Verified vs. Allocated Emissions}

\begin{center}
\includegraphics[width=0.85\textwidth]{verified_vs_allocated.png}
\end{center}

\vspace{-0.2cm}
\begin{itemize}
    \item \textbf{Teal}: Verified emissions; \textbf{Yellow dashed}: Free allocation
    \item Gap above allocation line = shortfall requiring market purchases
    \item Within-facility variation over time provides identification
\end{itemize}

\end{frame}

%% ========================================================================
%% SLIDE 5: GEOGRAPHIC DISTRIBUTION
%% ========================================================================
\begin{frame}{Geographic Distribution of Sample}

\begin{center}
\includegraphics[width=0.50\textwidth]{NUTS_heatmap.png}
\end{center}

\vspace{-0.2cm}
\begin{itemize}
    \item 521 facilities across 82 NUTS2 regions
    \item Concentration in Germany, Poland, Spain, France
    \item NUTS2 regions used for clustering and Region$\times$Year FE
\end{itemize}

\end{frame}

%% ========================================================================
%% SLIDE 6: DATA SOURCES
%% ========================================================================
\begin{frame}{Data Sources \& Sample Construction}

\textbf{Three Linked Data Sources:}
\begin{enumerate}
    \item \textbf{EEA Large Combustion Plant Registry}: Coordinates, capacity, fuel mix
    \item \textbf{EU ETS (EUTL)}: Verified emissions, free allocations
    \item \textbf{TROPOMI Satellite}: Daily NO$_2$ tropospheric columns (3.5$\times$5.5 km)
\end{enumerate}

\vspace{0.3cm}
\textbf{Sample Attrition:}
\begin{itemize}
    \item 3,405 plants $\rightarrow$ 932 facilities (500m spatial clustering)
    \item $\rightarrow$ 521 facilities with ETS linkage \& $\geq$3 years data
    \item $\rightarrow$ 291 facilities for satellite NOx ($\geq$20 valid days/year)
\end{itemize}

\vspace{0.3cm}
\textbf{Key Statistics:} Mean emissions 580 ktCO$_2$/yr, 80.8\% electricity sector

\end{frame}

%% ========================================================================
%% SLIDE 7: BEIRLE NOx METHODOLOGY
%% ========================================================================
\begin{frame}{Satellite NOx: Beirle Flux-Divergence Method}

\begin{columns}
\begin{column}{0.55\textwidth}
\textbf{Physical Basis:} Continuity equation
\[
\underbrace{\mathbf{w} \cdot \nabla V}_{\text{Advection}} \approx E - S
\]

\textbf{Implementation:}
\begin{enumerate}
    \item Spatial gradient of NO$_2$ (TROPOMI)
    \item $\times$ wind velocity (ERA5-Land)
    \item Integrate over 15 km disc
    \item Lifetime \& topographic corrections
\end{enumerate}

\vspace{0.2cm}
\textbf{Key Thresholds:}
\begin{itemize}
    \item Detection: 0.11 kg/s (conservative)
    \item Uncertainty: $\sim$35--45\%
\end{itemize}
\end{column}
\begin{column}{0.45\textwidth}
\includegraphics[width=\textwidth]{beirle_EDA.png}
\end{column}
\end{columns}

\end{frame}

%% ========================================================================
%% SLIDE 8: ALPHAEARTH EMBEDDINGS
%% ========================================================================
\begin{frame}{AlphaEarth Embeddings: High-Dimensional Controls}

\textbf{Google AlphaEarth Foundations} (2025): 64-dim geospatial representations learned from:
\begin{itemize}
    \item Multi-source satellite imagery (Sentinel-1/2, Landsat)
    \item Climate reanalysis (ERA5-Land), topography (GLO-30)
    \item Geotagged text (Wikipedia, GBIF)
\end{itemize}

\vspace{0.3cm}
\textbf{What Embeddings Encode:}
\begin{itemize}
    \item Land use context (urban density, industrial areas)
    \item Infrastructure (roads, built environment)
    \item Vegetation, climate patterns, terrain
\end{itemize}

\vspace{0.3cm}
\textbf{Why Use Them?}
\begin{itemize}
    \item Control satellite retrieval confounders (terrain $\rightarrow$ AMF; urban $\rightarrow$ background NO$_2$)
    \item High-dimensional spatial confounders that would be impractical to specify manually
    \item Extends Veitch et al. (2019) text embeddings to geospatial domain
\end{itemize}

\end{frame}

%% ========================================================================
%% SLIDE 9: EMBEDDING DIMENSIONALITY REDUCTION
%% ========================================================================
\begin{frame}{Embedding Dimensionality Reduction: PCA vs. PLS}

\textbf{Problem:} 64 dimensions may cause overfitting with limited within-facility variation

\vspace{0.3cm}
\textbf{PCA (Unsupervised):}
\begin{itemize}
    \item Projects onto directions maximizing variance in embedding space
    \item \textbf{Causally safe}---does not use outcome information
\end{itemize}

\vspace{0.3cm}
\textbf{PLS (Supervised):}
\begin{itemize}
    \item Projects onto directions that predict NOx outcome
    \item \textbf{Risk:} Naive PLS on panel $\rightarrow$ regularization bias (outcome snooping)
\item \textbf{Solution:} Train on \textit{facility-level means} (cross-sectional), not panel obs.
Then reproject every fac-year's embedding using the trained projector.
\end{itemize}

\vspace{0.3cm}
\textbf{Key Design:} PLS trained on $\bar{Y}_i^{\text{NOx}} = T^{-1}\sum_t Y_{it}$ (one obs/facility)
\begin{itemize}
    \item Resulting embeddings are \textit{time-invariant} within facility
    \item $\equiv$ pre-treatment covariates $\rightarrow$ no ``bad controls'' problem
    \item Both methods: 64-dim $\rightarrow$ 10 components
\end{itemize}

\end{frame}

%% ========================================================================
%% SLIDE 10: EMPIRICAL STRATEGY
%% ========================================================================
\begin{frame}{Empirical Strategy}

\textbf{Two-Way Fixed Effects (TWFE):}
\[
Y_{it} = \alpha_i + \gamma_{r(i),t} + \beta R_{it} + \mathbf{X}_{it}'\boldsymbol{\delta} + \varepsilon_{it}
\]

\vspace{0.2cm}
\begin{itemize}
    \item $\alpha_i$: Facility FE (absorb time-invariant confounders)
    \item $\gamma_{r(i),t}$: NUTS2 Region $\times$ Year FE (absorb regional trends)
    \item $\mathbf{X}_{it}$: Capacity, fuel shares, AlphaEarth embeddings (NOx only)
\end{itemize}

\vspace{0.3cm}
\textbf{Identification:} Within-facility variation in allocation ratios over time

\vspace{0.3cm}
\textbf{Key Design Choices:}
\begin{itemize}
    \item Cluster SEs by NUTS2 region (82 clusters)
    \item AlphaEarth embeddings (64-dim $\rightarrow$ 10-dim via PCA/PLS) control satellite confounders
    \item Do \textit{not} control for dispatch---mediator, not confounder
\end{itemize}

\end{frame}

%% ========================================================================
%% SLIDE 11: EMBEDDING CORRELATIONS
%% ========================================================================
\begin{frame}{AlphaEarth Embeddings: Component Correlations}

\begin{columns}
\begin{column}{0.5\textwidth}
\begin{center}
\textbf{PCA (Unsupervised)}
\includegraphics[width=\textwidth]{pca_component_correlations.png}
\end{center}
\end{column}
\begin{column}{0.5\textwidth}
\begin{center}
\textbf{PLS (Supervised)}
\includegraphics[width=\textwidth]{pls_component_correlations.png}
\end{center}
\end{column}
\end{columns}

\end{frame}

%% ========================================================================
%% SLIDE 12: EMBEDDING MAPS
%% ========================================================================
\begin{frame}{AlphaEarth Embeddings: Geographic Distribution}

\begin{columns}
\begin{column}{0.5\textwidth}
\begin{center}
\textbf{PCA Components 1--4}
\includegraphics[width=\textwidth]{pca_component_maps.png}
\end{center}
\end{column}
\begin{column}{0.5\textwidth}
\begin{center}
\textbf{PLS Components 1--4}
\includegraphics[width=\textwidth]{pls_component_maps.png}
\end{center}
\end{column}
\end{columns}

\end{frame}

%% ========================================================================
%% SLIDE 13: MAIN RESULTS - CO2
%% ========================================================================
\begin{frame}{Main Results: Verified CO$_2$ Emissions}

\begin{columns}
\begin{column}{0.5\textwidth}
\begin{table}[h]
\centering
\begin{tabular}{lc}
\toprule
& Log CO$_2$ \\
\midrule
Allocation Ratio & $-0.186$*** \\
& (0.030) \\
\midrule
95\% CI & [$-0.25$, $-0.13$] \\
Observations & 2,723 \\
Facilities & 521 \\
\bottomrule
\end{tabular}
\end{table}

\vspace{0.2cm}
\textbf{Interpretation:}
\begin{itemize}
    \item 10\% shortfall $\rightarrow$ \textbf{1.9\%} lower emissions
    \item At mean: $\sim$11 ktCO$_2$ reduction
    \item $p < 0.001$
\end{itemize}
\end{column}
\begin{column}{0.5\textwidth}
\includegraphics[width=\textwidth]{emissions_allocations_sectors.png}
\vspace{-0.3cm}
\begin{center}
\scriptsize Electricity (orange) vs. Other sectors (blue)
\end{center}
\end{column}
\end{columns}

\end{frame}

%% ========================================================================
%% SLIDE 14: MAIN RESULTS - NOx
%% ========================================================================
\begin{frame}{Main Results: Satellite-Derived NOx}

\begin{table}[h]
\centering
\small
\begin{tabular}{lcccc}
\toprule
& \multicolumn{2}{c}{DL $\geq$ 0.03 kg/s} & \multicolumn{2}{c}{DL $\geq$ 0.11 kg/s} \\
& PCA & PLS & PCA & PLS \\
\midrule
Allocation Ratio & $-0.000$ & $-0.000$ & $-0.003$** & $-0.003$*** \\
& (0.000) & (0.000) & (0.001) & (0.000) \\
\midrule
$p$-value & 0.84 & 0.65 & 0.017 & 0.001 \\
N & 577 & 577 & 140 & 140 \\
\bottomrule
\end{tabular}
\end{table}

\vspace{0.2cm}
\textbf{Key Pattern:}
\begin{itemize}
    \item \textbf{Null} at permissive threshold---noise dominates signal
    \item \textbf{Significant} at conservative threshold (DL $\geq$ 0.11 kg/s)
    \item 10\% shortfall $\rightarrow$ $\sim$\textbf{1.7\%} lower NOx (consistent with 1.9\% CO$_2$!)
\end{itemize}

\vspace{0.2cm}
\textbf{Cross-Validation:} Directional agreement between administrative \& satellite outcomes

\end{frame}

%% ========================================================================
%% SLIDE 15: CO2 HETEROGENEITY
%% ========================================================================
\begin{frame}{Heterogeneity: ETS CO$_2$ by Fuel, Sector, Location}

\begin{columns}
\begin{column}{0.55\textwidth}
\begin{table}[h]
\centering
\scriptsize
\begin{tabular}{llcc}
\toprule
Dimension & Group & $\beta$ & N \\
\midrule
\textit{Fuel} & Coal & $-0.98$*** & 653 \\
& Gas & $-0.21$*** & 1,197 \\
& Oil & $-0.45$ & 129 \\
& Biomass & $-0.11$*** & 438 \\
\midrule
\textit{Sector} & Electricity & $-0.21$*** & 2,173 \\
& Other & $-0.09$*** & 443 \\
\midrule
\textit{Location} & Rural & $-0.24$*** & 650 \\
& Urban & $-0.15$*** & 1,943 \\
\bottomrule
\end{tabular}
\end{table}
\end{column}
\begin{column}{0.45\textwidth}
\includegraphics[width=\textwidth]{average_fuel_mix.png}
\vspace{-0.3cm}
\begin{center}
\scriptsize Coal declining, gas/biomass rising
\end{center}
\end{column}
\end{columns}

\vspace{0.2cm}
\textbf{Interpretation:} Coal ($\sim$95 tCO$_2$/TJ) faces highest carbon intensity $\rightarrow$ strongest response. Electricity lost free allocation $\rightarrow$ full marginal cost.

\end{frame}

%% ========================================================================
%% SLIDE 16: CO2 HETEROGENEITY - COUNTRY & PYPSA
%% ========================================================================
\begin{frame}{Heterogeneity: ETS CO$_2$ by Country \& PyPSA Cluster}

\begin{columns}
\begin{column}{0.5\textwidth}
\textbf{By Country:}
\begin{table}[h]
\centering
\scriptsize
\begin{tabular}{lcc}
\toprule
Country & $\beta$ & N \\
\midrule
Spain & $-1.25$*** & 125 \\
Poland & $-0.33$*** & 743 \\
France & $-0.25$*** & 920 \\
Austria & $-0.24$*** & 227 \\
Sweden & $-0.11$*** & 421 \\
\bottomrule
\end{tabular}
\end{table}
\end{column}
\begin{column}{0.5\textwidth}
\textbf{By PyPSA-Eur Cluster:}
\begin{table}[h]
\centering
\scriptsize
\begin{tabular}{lcc}
\toprule
Cluster & $\beta$ & N \\
\midrule
PL0 2 (Poland) & $-1.46$*** & 228 \\
PL0 0 (Poland) & $-1.09$*** & 122 \\
PL0 1 (Poland) & $-0.38$* & 148 \\
AT0 0 (Austria) & $-0.26$** & 129 \\
FR0 9 (France) & $-0.25$** & 212 \\
\bottomrule
\end{tabular}
\end{table}
\end{column}
\end{columns}

\vspace{0.2cm}
\textbf{Interpretation:} Polish clusters show strongest effects (5--7$\times$ pooled estimate)---coal-heavy generation mix. PyPSA clusters group facilities facing correlated prices \& dispatch.

\end{frame}

%% ========================================================================
%% SLIDE 17: NOx HETEROGENEITY
%% ========================================================================
\begin{frame}{Heterogeneity: Satellite NOx (DL $\geq$ 0.11, PLS)}

\begin{table}[h]
\centering
\small
\begin{tabular}{llccc}
\toprule
Dimension & Group & $\beta$ & $p$ & N \\
\midrule
\textit{Sector} & Electricity & $-0.003$*** & .002 & 132 \\
\textit{Interference} & Yes ($<$20km) & $-0.003$** & .013 & 140 \\
\textit{Stat Error} & Low ($<$30\%) & $-0.003$** & .013 & 140 \\
\midrule
\textit{Location} & Urban & $+0.007$** & .045 & 128 \\
\textit{Fuel} & Gas & $-0.003$ & .90 & 68 \\
& Coal & $+0.051$*** & .001 & 38 \\
\textit{Country} & France & $-0.004$*** & .008 & 96 \\
& Poland & $+0.079$** & .021 & 44 \\
\bottomrule
\end{tabular}
\end{table}

\vspace{0.2cm}
\textbf{Findings:} Electricity drives main effect; robustness to interference \& stat error. \\
\textbf{Anomalies:} Urban/Coal/Poland show positive $\beta$---likely measurement noise (small N=140).

\end{frame}

%% ========================================================================
%% SLIDE 18: ROBUSTNESS
%% ========================================================================
\begin{frame}{Robustness Summary}

\textbf{Cross-Outcome Consistency:}
\begin{itemize}
    \item Both CO$_2$ and NOx show negative effects (1.9\% vs. 1.7\%)
    \item Agreement across independent measurement systems
\end{itemize}

\vspace{0.3cm}
\textbf{Embedding Method Stability:}
\begin{itemize}
    \item PCA vs. PLS nearly identical at DL $\geq$ 0.11 ($-0.00282$ vs. $-0.00301$)
    \item Not sensitive to dimensionality reduction approach
\end{itemize}

\vspace{0.3cm}
\textbf{Measurement Quality:}
\begin{itemize}
    \item \textbf{Spatial interference}: Effects persist for interfered facilities ($\beta = -0.003$**, $p = 0.013$)
    \item \textbf{Detection limit}: Null $\rightarrow$ significant pattern consistent with physics
    \item Inverse-variance weighting accounts for heteroskedasticity
\end{itemize}

\end{frame}

%% ========================================================================
%% SLIDE 19: CONTRIBUTIONS & CONCLUSION
%% ========================================================================
\begin{frame}{Contributions \& Conclusion}

\textbf{Three Methodological Contributions:}
\begin{enumerate}
    \item \textbf{AlphaEarth Embeddings}: Geospatial foundation model as high-dim controls (extends Veitch et al. 2019 to geospatial domain)
    \item \textbf{NUTS2 + PyPSA-Eur}: Administrative regions for inference, network clusters for electricity heterogeneity
    \item \textbf{Beirle Flux-Divergence}: Atmospheric physics adapted for panel econometrics
\end{enumerate}

\vspace{0.3cm}
\textbf{Key Empirical Findings:}
\begin{itemize}
    \item 10\% allocation shortfall $\rightarrow$ \textbf{1.9\% lower CO$_2$} ($p < 0.001$)
    \item Satellite NOx corroborates at conservative DL ($\sim$1.7\%)
    \item Effects strongest for \textbf{coal-dominant} and \textbf{electricity-sector} facilities
\end{itemize}

\vspace{0.3cm}
\textbf{Broader Impact:} Framework for dual-outcome policy evaluation combining administrative records with satellite remote sensing

\end{frame}

\end{document}
