\documentclass[aspectratio=169]{beamer}
\usetheme{metropolis}       
\usepackage{graphicx, booktabs}
\usepackage{amsmath, amssymb}
\usepackage{hyperref}
\hypersetup{colorlinks=true, urlcolor=blue}

\title[Policy Effects on Site-Level Emissions]{Policy Effects on Site-Level Emissions:\break Master's Thesis Initial Proposal}
\author{Arnav Agarwal}
\institute{QMSS, Columbia University}
\date{October 2025}

% ---------- helper: figure placeholder box ----------
\newcommand{\placeholder}[2][0.85\linewidth]{%
  \begin{center}
    \fbox{\parbox{#1}{\centering\vspace{2.5em}\textbf{PLACEHOLDER}\\[0.3em]#2\\[2.5em]}}
  \end{center}
}

\begin{document}
\frame{\titlepage}

% -------------------- Motivation --------------------
\begin{frame}{Motivation}
    \begin{itemize}
      \item Environmental policies are designed to cut pollution — but do they make a clear, local difference around major industrial sites?
      \item A site-level view keeps the focus on the places where emissions actually occur and where communities are affected.
      \item New satellite data now lets us track air quality consistently across countries and over time.
\item High quality AI-derived embeddings also allow for encoding of high-dimensional geospatial controls.
      \item We follow plants before and after policy rules take effect to see how local air quality changes.
      \item Clear evidence helps governments and industry target efforts, improve enforcement, and prioritise investments where they matter most.
    \end{itemize}
    \end{frame}

% -------------------- Research Questions --------------------
\begin{frame}{Research Questions}
\begin{itemize}
  \item \textbf{RQ1: Average effect.} What is the effect of policy exposure on tropospheric NO\textsubscript{2} within buffers around treated facilities?
  \item \textbf{RQ2: Heterogeneity.} How do effects vary by primary fuel, capacity, and controls (SCR/SNCR/FGD), and by local context?
  \item \textbf{RQ3: Moderation.} How do seasonal and meteorological conditions (winds, PBLH) moderate short–run impacts?
  \item \textbf{RQ4: Dynamics.} Event–time profile (anticipation, persistence).
\end{itemize}
\end{frame}

% -------------------- Causal Diagram / Variables --------------------
\begin{frame}{Causal Diagram \& Variables}
\begin{columns}[T,onlytextwidth]
\column{0.44\textwidth}
\textbf{Key variables}
\begin{itemize}
  \item $P_{it}$: policy exposure (US EPA EASEY Service; EU ETS) at facility $i$, month $t$.
  \item $E_{it}$: NO\textsubscript{2} around facility (e.g., 10\,km buffer from TROPOMI).
  \item $X_i$: plant traits (capacity, primary fuel); abatement 
  \item $G_{it}$: geographic/land-use embeddings around the site.
  \item $W_{it}$: weather/transport (winds, PBLH, temperature, precip.).
\end{itemize}
\medskip
\column{0.55\textwidth}
\begin{figure}[h]
\centering
\includegraphics[width=1.1\textwidth]{dag.png}
\end{figure}
\end{columns}
\end{frame}

% -------------------- Data Sources --------------------
\begin{frame}{Data Sources}
    \footnotesize
    \begin{columns}[T,onlytextwidth]
    \column{0.5\textwidth}
    \textbf{Plants \& geometry}
    \begin{itemize}\setlength{\itemsep}{2pt}
      \item EIA–860 (US, annual): capacity, fuel, coords.
      \item EIA–860M (US, monthly): operability mask.
      \item EPA FRS (US, static): backup geocodes.
      \item EEA LCP (EU, annual): plant attrs \& coords.
    \end{itemize}
    
    \textbf{Policy / exposure}
    \begin{itemize}\setlength{\itemsep}{2pt}
      \item US EPA EASEY Service: policy x time x location flags under various US government programmes; ozone season.
      \item EU ETS (Energy Trading System): credits exposure.
    \end{itemize}
    
    \column{0.5\textwidth}
    \textbf{Outcomes \& controls}
    \begin{itemize}\setlength{\itemsep}{2pt}
      \item TROPOMI NO\textsubscript{2} (D$\to$M): buffered columns.
      \item ERA5 (monthly): winds, PBLH, temp, precip.
      \item AlphaEarth (annual): geographic/context embeddings.
    \end{itemize}
    \end{columns}
    \end{frame}

% -------------------- Methodology  --------------------
\begin{frame}{Methodology (Post-Data Processing)}
    \textbf{1) Feature engineering}
    \begin{itemize}\setlength{\itemsep}{3pt}
      \item Outcome: satellite NO\textsubscript{2} around each facility (monthly).
      \item Exposure: monthly policy indicators; operability status.
      \item Covariates: core plant traits; weather/climate \emph{embeddings} capturing spatiotemporal variation (as in prior MLC work).
    \end{itemize}

    \textbf{2) Identification}
    \begin{itemize}\setlength{\itemsep}{3pt}
      \item Staggered policy timing $\Rightarrow$ event–study/DiD with unit and time effects.
    \end{itemize}
    
    \textbf{3) Estimation}
    \begin{itemize}\setlength{\itemsep}{3pt}
      \item Perform estimation with simpler (e.g linear) to complex (e.g. deep learning) models.
      \item Probe ATE and CATEs (fuel, size, controls, context) and run robustness checks.
    \end{itemize}
    \end{frame}

% -------------------- Quick Example of a Spatiotemporal Encoder ---------
\begin{frame}{Example Spatiotemporal Encoder: Conv-LSTM}
\begin{itemize}
\item For capturing spatiotemporal variation in weather/climate variables we can use encoders like
Conv-LSTMs (or Conv3D, Temporal Convolutional Networks, etc.) and get vectorised embeddings.
\end{itemize}
\begin{figure}[h]
\centering
\includegraphics[width=0.7\textwidth]{encoder_example.png}
\end{figure}
\end{frame}
 

% -------------------- Initial EDA: Facility Universe --------------------
\begin{frame}{Initial EDA: Facility Universe (US \& EU)}
\begin{itemize}
  \item Construct one row per plant with \{lat, lon, capacity\_MW, fuel\_primary, abatement flags\}.
  \item Merge US EIA–860 and EU LCP harmonised fields into a common schema.
\end{itemize}
\medskip
\begin{figure}[h]
\centering
\includegraphics[width=0.5\textwidth]{capacity_hist.png}
\end{figure}
\end{frame}

\begin{frame}{Initial EDA: Facility Universe (US \& EU)}
\begin{figure}[h]
\centering
\includegraphics[width=0.25\textwidth]{fuel_composition.png}
\end{figure}
\end{frame}

% -------------------- Initial EDA: Policy Panel --------------------
\begin{frame}{Initial EDA: Policy Panel \& Timing}
\begin{itemize}
  \item Monthly flags: \texttt{treat\_any}, \texttt{treat\_cap}; ozone–season logic for CSAPR NO\textsubscript{x} OS (May–Sept).
  \item Mask US exposures using EIA–860M operability where available (leave EU unchanged).
  \item Shares are computed over operable rows when coverage exists.
\end{itemize}
\medskip
\begin{figure}[h]
\centering
\includegraphics[width=0.5\textwidth]{share_treated.png}
\end{figure}
\medskip
\end{frame}

% -------------------- Challenges --------------------
\begin{frame}{Key Challenges (Probably Scope) \& Design Choices}
\textbf{Linkage \& scope}
\begin{itemize}
  \item EU ETS installations $\neq$ EEA–LCP IDs $\Rightarrow$ fuzzy + geo join (\(\pm\)1–2\,km).
\item Various data sources, fused using multiple methods (combining deep learning and standard
estimation methods), can by data-hungry and can be prone to instability
\end{itemize}
\textbf{Measurement}
\begin{itemize}
  \item Satellite QA/cloud filtering; buffer–size sensitivity (5–20 km); urban background vs rural.
\end{itemize}
\end{frame}

\begin{frame}{Key Challenges (Probably Scope) \& Design Choices}
    \textbf{Identification}
    \begin{itemize}
    \item Time-Dependent Treatment Effects (TDTTE): Determining timescales for when we expect the full
    effect of out treatments to have manifested. Similarly, anticipation windows / placebo leads.
    \item Although capturing policy attributes alleviates, issue with encoding varying treatment features.
    \item Spatial spillovers (e.g plants within close geographical proximity resulting in independence violation).
    \end{itemize}
\end{frame}

\end{document}